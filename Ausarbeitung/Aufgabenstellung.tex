\section{Aufgabenstellung}
	\subsection{Teil 1}
		\begin{abstract}
			\noindent Zitieren Sie \textbf{schriftlich}(kurz) mit Hilfe \underline{wissenschaftlicher Literatur} folgende relevanten
			Begrifflichkeiten:
			\begin{enumerate}
				\item Kundenorientierung
				\item Zielgruppe
				\item Kundenzufriedenheit
				\item Kundennutzen
				\item Kundenvorteil
				\item Marketingstrategie
				\item Produktpolitik
				\item Preispolitik
				\item Vertriebspolitik
				\item Kommunikationspolitik
			\end{enumerate}
			\textbf{Belegen} und \textbf{zitieren} (\underline{direkte Zitierweise}) Sie jeden Begriff mit \underline{mindestens}
			\textbf{zwei verschiedenen wissenschaftlichen Primärquellen} (kein Wikipedia, Vorlesungsskript, Lexika etc.) entsprechend den
			wissenschaftlichen Vorgaben des Fachgebietes Marketing. (siehe \url{https://bit.ly/3UJYT7c})
			\newline
			Erläutern Sie zusätzlich jeden Begriff kurz mit eigenen Worten und anhand eines selbst gewählten Beispiels!
		\end{abstract}
	\subsection{Teil 2}
		\begin{itemize}
			\item Entwickeln Sie einen \textbf{Namen} für Ihr neu zu gründendes Unternehmen sowie ein \textbf{Logo}
			bzw. einen \textbf{Slogan!}
			\item Charakterisieren Sie die \textbf{Art der Leistung}, die Sie anbieten wollen!
			\item Erfassen Sie kurz die für Ihr Unternehmen relevanten Faktoren (mit relevanten
			Zahlenbeispielen, Fakten, Studien etc.) im Rahmen der \textbf{globalen Umfeldanalyse!}
			\item Charakterisieren Sie daraufhin den \textbf{Markt} (räumliche, geografische
			Marktabgrenzung), in dem Ihr Unternehmen tätig sein soll!
			\item Benennen Sie Ihre \textbf{Zielgruppen} und zeigen Sie auf, welchen \textbf{Kundennutzen} Sie
			schaffen möchten!
			\item Gehen Sie auf \textbf{Wettbewerber} (zwei bis drei) ein und charakterisieren Sie Ihren
			\textbf{Wettbewerbsvorteil!}
		\end{itemize}
		\newpage
	\subsection{Teil 3}
		\begin{abstract}
			\noindent Erläutern Sie für Ihr Unternehmen die \textbf{Ausgestaltung des Marketing-Mix}, d.h. der
			\begin{itemize}
				\item Produktpolitik,
				\item Preispolitik,
				\item Vertriebspolitik und der
				\item Kommunikationspolitik
			\end{itemize} 
			Leiten Sie Ihre \textbf{Schlussfolgerung} auf \textbf{Basis} der erarbeiteten Sachverhalte aus der \textbf{ersten Präsentation} ab!
		\end{abstract}
		\newpage