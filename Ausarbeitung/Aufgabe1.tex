\section{Teil 1}
    \subsection{Kundenorientierung}
       \subsubsection*{1. Quelle:}
        \begin{abstract}
        \noindent„Kundenorientierung ist die umfassende, kontinuier-
liche Ermittlung und Analyse der individuellen Kundenerwartungen sowie deren interne
und externe Umsetzung in unternehmerische Leistungen sowie Interaktionen im Rahmen
eines Relationship-Marketing-Konzeptes mit dem Ziel, langfristig stabile und öko-
nomisch vorteilhafte Kundenbeziehungen zu etablieren.“ (Bruhn, 2016a)
            %1. Quelle Kundenorientierung
        \end{abstract}
        \subsubsection*{2. Quelle:}
        \begin{abstract}
        \noindent"Kundenorientierung ist eine Denkhaltung sowie ein Managementmodell, das
aus vier unterschiedlichen Dimensionen besteht. 1) Customer Value-based Decision
Making: Die Verantwortlichen sollen Entscheidungen so treffen, dass der Customer
Value (Customer-Firm Value) kontinuierlich steigt. 2) Customer-centric Transformation:
Kundenorientierung ist als kontinuierlicher Transformationsprozess zu verstehen, der die
Reagibilität auf mögliche Veränderungen der Einstellung und des Verhaltens der Kunden
absichert. 3) Co-Creation: Im Kern dient die Kundenorientierung dazu, den Kunden
möglichst wertstiftend in die Organisation zu integrieren. 4) Customer Management: Im
Ergebnis wird die Organisation befähigt, ein differenzierendes Kundenmanagement zu
etablieren, das wertvollere Beziehungen zu den Kunden auf- und ausbauen kann als der
Wettbewerb." (Staudacher, 2021 S.21 )
            %2. Quelle Kundenorientierung
        \end{abstract}
        \subsubsection*{Eigene Erläuterung:}
        \begin{abstract}
            %Eigene Erläuterung Kundenorientierung
        \end{abstract}
    \subsection{Zielgruppe}
        \subsubsection*{1. Quelle:}
        \begin{abstract}
        \noindent Um eine Strategie im Rahmen der Kommunikationspolitik möglichst präg-
nant gestalten und diese gezielt ausrichten zu können, bedarf es eines Seg-
mentbezuges. Es sollten also Zielgruppen gebildet werden, die homogener
auf entsprechende kommunikationspolitische Maßnahmen reagieren als der
Gesamtmarkt. Eine derartige Zielgruppenabgrenzung kann z. B. nach demo-
grafischen, geografischen oder psychografischen Kriterien oder mit Blick auf das beobachtbare Verhalten erfolgen. Als Zielgruppe der Kommunikations-
politik kann jede Art von Anspruchsgruppe definiert werden, z. B. Konsu-
menten, Käufer, Verwender, Großhändler, Einzelhändler oder auch Mei-
nungsführer.184 Bei der Ansprache mehrerer Zielgruppen ist auf eventuelle
Zielkonflikte zu achten. (Olbrich, 2022 S. 196)
            %1. Quelle Zielgruppe
        \end{abstract}
        \subsubsection*{2. Quelle:}
        \begin{abstract}
            %2. Quelle Zielgruppe
        \end{abstract}
        \subsubsection*{Eigene Erläuterung:}
        \begin{abstract}
            %Eigene Erläuterung Zielgruppe
        \end{abstract}
    \subsection{Kundenzufriedenheit}
        \subsubsection*{1. Quelle:}
        \begin{abstract}
            \noindent Kundenzufriedenheit ist, dass immer kleinere Kundensegmente
            mit (relativ) ähnlichen Bedürfnissen gebildet werden. Auf diese Weise führt
            der Marktwandel in Verbindung mit der Zielkomplexität zur Kundenkomple-
            xität. (Olbrich, 2022 S. 129)
            %1. Quelle Kundenzufriedenheit
        \end{abstract}
        \subsubsection*{2. Quelle:}
        \begin{abstract}
            %2. Quelle Kundenzufriedenheit
        \end{abstract}
        \subsubsection*{Eigene Erläuterung:}
        \begin{abstract}
            %Eigene Erläuterung Kundenzufriedenheit
        \end{abstract}
    \subsection{Kundennutzen}
        \subsubsection*{1. Quelle:}
        \begin{abstract}
            %1. Quelle Kundennutzen
        \end{abstract}
        \subsubsection*{2. Quelle:}
        \begin{abstract}
            %2. Quelle Kundennutzen
        \end{abstract}
        \subsubsection*{Eigene Erläuterung:}
        \begin{abstract}
            %Eigene Erläuterung Kundennutzen
        \end{abstract}
    \subsection{Kundenvorteil}
        \subsubsection*{1. Quelle:}
        \begin{abstract}
            \noindent Die Effektivität wird im Rahmen dieser Perspektive als ein externes Leis-
            tungsmaß angesehen, das angibt, ob ein Unternehmen den Erwartungen und
            Ansprüchen der Kunden gerecht wird. Die Effizienz soll hingegen ein inter-
            nes Leistungsmaß darstellen. Es soll das Verhältnis zwischen Input und Out-
            put angeben. Effektivität und Effizienz stellen im Rahmen dieser Betrachtung
            zwei Komponenten eines Wettbewerbsvorteils dar. Diese Zweidimensionali- Komponenten eines
            tät des Wettbewerbsvorteils soll zeigen, dass bei der Orientierung eines An- Werttbewerbsvorteils
            bieters im Wettbewerb zwei Richtungen der Vorteilsfindung getrennt werden
            müssen: Die anbieterexterne Sphäre und die anbieterinterne Sphäre. Vorteile,
            die aus der externen Sphäre resultieren, werden als Kundenvorteile bezeich-
            net. Vorteile, die aus der internen Sphäre resultieren, können als Anbieter-
            vorteile bezeichnet werden.400 Natürlich gibt es neben dieser Definition für
            das Konstrukt des Wettbewerbsvorteils auch eine Vielzahl anderer Definiti-
            onen.401 Die hier getroffene Abgrenzung erscheint jedoch für die Betrachtung
            von Investitionsgütermärkten von besonderer Bedeutung. (Olbrich 2022 S. 393)
            %1. Quelle Kundenvorteil
        \end{abstract}
        \subsubsection*{2. Quelle:}
        \begin{abstract}
            %2. Quelle Kundenvorteil
        \end{abstract}
        \subsubsection*{Eigene Erläuterung:}
        \begin{abstract}
            %Eigene Erläuterung Kundenvorteil
        \end{abstract}
    \subsection{Marketingstrategie}
        \subsubsection*{1. Quelle:}
        \begin{abstract}
            \noindent "Unter Marketingstrategien versteht man Hand-
            lungsprogramme zur Erreichung von bestimmten Zielen.14 Entsprechende
            Strategien berücksichtigen die Wettbewerbssituation, die Bedürfnisse der
            Nachfrager und das bisherige Angebot des Unternehmens. Sie führen unter
            Heranziehung von Prognosen hinsichtlich veränderlicher Umweltgrößen zu
            einer konkreten Ausprägung der Marketinginstrumente." (Olbrich, 2022 S. 21)
            %1. Quelle Marketingstrategie
        \end{abstract}
        \subsubsection*{2. Quelle:}
        \begin{abstract}
            %2. Quelle Marketingstrategie
        \end{abstract}
        \subsubsection*{Eigene Erläuterung:}
        \begin{abstract}
            %Eigene Erläuterung Marketingstrategie
        \end{abstract}
    \subsection{Produktpolitik}
        \subsubsection*{1. Quelle:}
        \begin{abstract}
            %1. Quelle Produktpolitik
            
        \end{abstract}
        \subsubsection*{2. Quelle:}
        \begin{abstract}
            %2. Quelle Produktpolitik
        \end{abstract}
        \subsubsection*{Eigene Erläuterung:}
        \begin{abstract}
            %Eigene Erläuterung Produktpolitik
        \end{abstract}
    \subsection{Preispolitik}
        \subsubsection*{1. Quelle:}
        \begin{abstract}
            %1. Quelle Preispolitik
            \noindent "Die Preispolitik als Teilbereich des Marketing stellt somit nicht ein
            isoliertes Entscheidungsfeld dar, sondern muss im Kontext des gesamten un-
            ternehmerischen Handelns gesehen werden. Es bestehen z. B. Interdependen-
            zen zu den Bereichen Produktion und Finanzierung. In der Produktion muss
            die Kapazitätsplanung mit der Preispolitik koordiniert werden. Mit Blick auf
            die Finanzierung ist ein Preiskampf nur dann durchführbar, wenn die Zah-
            lungsfähigkeit des Unternehmens sichergestellt werden kann.Die Preispolitik betrifft allerdings nicht nur Fragen über die Höhe des Preises,
            es muss auch über die Form der Preissetzung entschieden werden. In der Pra- Formen der
            xis können neben sogenannten linearen Preisen (fester Verkaufspreis pro Preissetzung
            Mengeneinheit) z. B. auch nicht-lineare Tarife und Preisbündelungen beo-
            bachtet werden. Nicht-lineare Tarife beinhalten nach Verkaufsmengen ge-
            staffelte Preise oder eine Teilung des Preises in eine Grundgebühr und ein
            mengenabhängiges Entgelt – wie es in der Telekommunikationsbranche und
            bei vielen Versorgungsunternehmen üblich ist. Bietet ein Unternehmen eine
            Kombination von Produkten oder Dienstleistungen zu einem Preis an, so
            spricht man von Preisbündelung. I. d. R. verlangt das Unternehmen für dieses
            ‚Set‘ einen geringeren Preis als die Summe der Einzelpreise. Beide Maßnah-
            men sollen die Kunden dazu veranlassen, einen höheren Umsatz (pro Ge-
            schäftsvorfall) mit dem Unternehmen zu tätigen. Im ersten Fall (nicht-lineare
            Tarife) sollen die Kunden eine größere Menge, im zweiten Fall (Preisbünde-
            lung) weitere Produkte kaufen." (Olbrich, 2022 S. 157)
        \end{abstract}
        \subsubsection*{2. Quelle:}
        \begin{abstract}
            %2. Quelle Preispolitik
        \end{abstract}
        \subsubsection*{Eigene Erläuterung:}
        \begin{abstract}
            %Eigene Erläuterung Preispolitik
        \end{abstract}
    \subsection{Kommunikationspolitik}
        \subsubsection*{1. Quelle:}
        \begin{abstract}
            %1. Quelle Kommunikationspolitik
            \noindent "Allgemein umfasst die Kommunikationspolitik die Gestaltung der auf die
            Märkte gerichteten Informationen und der Informationskanäle. Bei vielen
            Verbrauchern ist – nicht zuletzt aufgrund der Nutzung der Kommunikations-
            politik zur Differenzierung und der angestiegenen Anzahl entsprechender
            Werbebotschaften in Printmedien, im Fernsehen und im Internet – ein stark
            nachlassendes Informationsinteresse zu konstatieren. Besonders im Rahmen
            der TV-Werbung ist ein ausgeprägtes Reaktanzverhalten (z. B. bei Werbe-
            einblendungen) festzustellen, das sich durch sogenanntes ‚Zapping’ (also ,Zapping‘
            durch Wechseln des Programmes) manifestiert. Allerdings lässt sich auch bei
            Auftreten eines derartigen Reaktanzverhaltens durch Anwendung des kom-
            munikationspolitischen Instrumentariums gegensteuern. So können Produkte
            im Rahmen des Product Placement z. B. in Spielfilmen, Shows und Spielen Product Placement
            platziert werden." (Olbrich, 2022 S. 193)
        \end{abstract}
        \subsubsection*{2. Quelle:}
        \begin{abstract}
            %2. Quelle Kommunikationspolitik
        \end{abstract}
        \subsubsection*{Eigene Erläuterung:}
        \begin{abstract}
            %Eigene Erläuterung Kommunikationspolitik
        \end{abstract}
    \subsection{Vertriebspolitik}
        \subsubsection*{1. Quelle:}
        \begin{abstract}
            %1. Quelle Vertriebspolitik
        \end{abstract}
        \subsubsection*{2. Quelle:}
        \begin{abstract}
            %2. Quelle Vertriebspolitik
        \end{abstract}
        \subsubsection*{Eigene Erläuterung:}
        \begin{abstract}
            %Eigene Erläuterung Vertriebspolitik
        \end{abstract}
    