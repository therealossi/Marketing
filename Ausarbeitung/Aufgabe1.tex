\section{Teil 1}
    \subsection{Kundenorientierung}
       \subsubsection*{1. Quelle:}
        \begin{abstract}
            %1. Quelle Kundenorientierung
            \noindent„Kundenorientierung ist die umfassende, kontinuierliche 
            Ermittlung und Analyse der individuellen Kundenerwartungen sowie deren interne
            und externe Umsetzung in unternehmerische Leistungen sowie Interaktionen im Rahmen
            eines Relationship-Marketing-Konzeptes mit dem Ziel, langfristig stabile und 
            ökonomisch vorteilhafte Kundenbeziehungen zu etablieren.“ \cite{Bruhn2016}
        \end{abstract}
        \subsubsection*{2. Quelle:}
        \begin{abstract}
            %2. Quelle Kundenorientierung
            \noindent "Kundenorientierung ist eine Denkhaltung sowie ein Managementmodell, das
            aus vier unterschiedlichen Dimensionen besteht. 1) Customer Value-based Decision
            Making: Die Verantwortlichen sollen Entscheidungen so treffen, dass der Customer
            Value (Customer-Firm Value) kontinuierlich steigt. 2) Customer-centric Transformation:
            Kundenorientierung ist als kontinuierlicher Transformationsprozess zu verstehen, der die
            Reagibilität auf mögliche Veränderungen der Einstellung und des Verhaltens der Kunden
            absichert. 3) Co-Creation: Im Kern dient die Kundenorientierung dazu, den Kunden
            möglichst wertstiftend in die Organisation zu integrieren. 4) Customer Management: Im
            Ergebnis wird die Organisation befähigt, ein differenzierendes Kundenmanagement zu
            etablieren, das wertvollere Beziehungen zu den Kunden auf- und ausbauen kann als der
            Wettbewerb." \cite{Staudacher2021}
        \end{abstract}
        \subsubsection*{Eigene Erläuterung:}
        \begin{abstract}
            %Eigene Erläuterung Kundenorientierung
            \noindent 
            Kundenorientierung ist die Integrierung der Kundenbedürfnisse in die eigene 
            Produktpolitik. Dabei wird eine Beziehung zu dem Kunden aufgebaut, die den Kunden
            stärker bindet, als die pure Produktleistung, die angeboten wird. Die genutzten Elemente
            gehen dabei also über die wirtschaftlichen Kenngrößen hinaus.
            Man könnte von den \"{}Soft Skills\"{} eines Unternehmens sprechen.
            \newline 
            Man stelle sich beispielsweise die Vermietung einer Ferienwohnung vor. Die Vermietung
            stellt nicht nur eine Ferienwohnung bereit. Sie informiert außerdem noch über 
            Veranstaltungen und Sehenswürdigkeiten in der Nähe. Sie steht für den Kunden jederzeit
            für alle Probleme zur Verfügung. Der Kunde bekommt Kleinigkeiten, wie das Vorbeibringen 
            von frischen Brötchen am Morgen oder das Vermeiden vom lauten Heckenschneiden um die Mittagszeit.
            Dazu wird durch nette Gespräche ein familiäres Umfeld aufgebaut, in dem der Kunde sich rundum
            wohlfühlt.
            \newline
            Dieses Gefühl der Wertschätzung und Integration bindet den Kunden längerfristiger als jegliche
            Art von Preisvorteil oder Sonstigem.
            \newline
            Der Kunde wird sich bei seinem nächsten Urlaub in der Nähe der Region gar nicht erst
            nach billigeren Alternativen umsehen, wenn die gewohnte noch verfügbar ist, denn es ist 
            nicht mehr nur die Wohneigenschaft, welche erfüllt werden muss, um mit der vorherigen Ferienwohnung zu konkurrieren.
        \end{abstract}
    \subsection{Zielgruppe}
        \subsubsection*{1. Quelle:}
        \begin{abstract}
            %1. Quelle Zielgruppe
            \noindent \"{}Um eine Strategie im Rahmen der Kommunikationspolitik möglichst 
            prägnant gestalten und diese gezielt ausrichten zu können, bedarf es eines 
            Segmentbezuges. Es sollten also Zielgruppen gebildet werden, die homogener
            auf entsprechende kommunikationspolitische Maßnahmen reagieren als der
            Gesamtmarkt. Eine derartige Zielgruppenabgrenzung kann z. B. nach 
            demografischen, geografischen oder psychografischen Kriterien oder mit 
            Blick auf das beobachtbare Verhalten erfolgen. Als Zielgruppe der 
            Kommunikationspolitik kann jede Art von Anspruchsgruppe definiert werden, z. B. 
            Konsumenten, Käufer, Verwender, Großhändler, Einzelhändler oder auch 
            Meinungsführer.$^{184}$ Bei der Ansprache mehrerer Zielgruppen ist auf eventuelle
            Zielkonflikte zu achten." \cite{Olbrich2022}
        \end{abstract}
        \subsubsection*{2. Quelle:}
        \begin{abstract}
            %2. Quelle Zielgruppe
            \noindent \"{}Zielgruppen sind nicht mit Marktsegmenten gleichzusetzen. Während bei der Markt-
            segmentierung aktuelle und potenzielle Käufer identifiziert werden, die durch den
            Einsatz der Marketinginstrumente differenziert zu bearbeiten sind, werden im Rahmen der
            werblichen Zielgruppenplanung jene Gruppen bestimmt, die durch die Werbung 
            angesprochen werden. Zielgruppen sind die mit einer Kommunikationsbotschaft anzusprechenden 
            Empfänger (Rezipienten) der Kommunikation. Hierbei sind nicht nur aktuelle und potenzielle Käufer von Interesse, sondern auch
            Gruppen, die einen Einfluss auf die Entscheidungen der Käufer ausüben wie z.B. 
            Meinungsführer oder Referenzpersonen." \cite{Bruhn2019}
        \end{abstract}
        \subsubsection*{Eigene Erläuterung:}
        \begin{abstract}
            %Eigene Erläuterung Zielgruppe
            \noindent Zielgruppen sind die Segmentierung von Gruppen, welche unterschiedlich auf die verschiedenen
            Instrumente des Marketings reagieren. Es ist die Beantwortung der Fragen: "Wer sieht mein Produkt wann, wo und warum? Fördert diese Präsentation mein Produkt?"
            \newline
            Man stelle sich Veranstalter vor, die eine Party für 20- bis 28-Jährige bewerben wollen.
            Logischerweise ist das Schalten einer Anzeige in der Zeitung dafür völlig unpassend.
            Nur wenige der 20- bis 28-Jährigen werden diese Anzeige überhaupt sehen und selbst die, 
            die sie sehen, werden den Zeitungsartikel nicht mit Partyspaß assoziieren. 
            Für das Bewerben des all jährigen Stadtfestes könnte dieser Zeitungsartikel aber genau richtig sein, 
            weil dabei andere Zielgruppen angesprochen werden sollen.
            \newline
            Für das Bewerben der Party wäre ein Beitrag in Social-Media viel geeigneter, da die gewünschte Zielgruppe dort mehr vertreten ist. Interesse an der Veranstaltung kann schnell 
            geteilt werden und für noch mehr Interesse sorgen. Insgesamt kann viel größere Aufmerksamkeit
            durch das Abstimmen auf Zielgruppen effizienter generiert werden. 
        \end{abstract}
        \newpage
    \subsection{Kundenzufriedenheit}
        \subsubsection*{1. Quelle:}
        \begin{abstract}
            %1. Quelle Kundenzufriedenheit
            \noindent "Kundenzufriedenheit ist, dass immer kleinere Kundensegmente
            mit (relativ) ähnlichen Bedürfnissen gebildet werden. Auf diese Weise führt
            der Marktwandel in Verbindung mit der Zielkomplexität zur Kundenkomplexität." \cite{Olbrich2022}
        \end{abstract}
        \subsubsection*{2. Quelle:}
        \begin{abstract}
            %2. Quelle Kundenzufriedenheit
            \noindent "Die Kernaussage des C/D-Paradigmas lautet, dass Kundenzufriedenheit aus
            dem Vergleich der tatsächlichen Erfahrung bei der Inanspruchnahme einer Leistung
            (Ist-Leistung) mit einem bestimmten Vergleichsstandard des Kunden (Soll-Leistung)
            resultiert. Entspricht die wahrgenommene Ist-Leistung der Soll-Leistung, so spricht
            man von Bestätigung (Confirmation). Das Zufriedenheitsniveau, das bei exakter
            Übereinstimmung der wahrgenommenen Leistung mit dem Vergleichsstandard vorliegt,
            bezeichnet man als Konfirmationsniveau der Zufriedenheit." \cite{Homburg2020}
        \end{abstract}
        \subsubsection*{Eigene Erläuterung:}
        \begin{abstract}
            %Eigene Erläuterung Kundenzufriedenheit
            \noindent Kundenzufriedenheit ist das Verhältnis zwischen der Erwartung des Kunden an das
            Produkt und der Erfüllung dieser durch dasselbe.
            \newline
            Ein Beispiel für Kundenzufriedenheit ist die Webplattform wish, die für 
            ihre billigen Produkte bekannt ist.
            Vermutlich werden dort viele Käufer wenig Kundenzufriedenheit empfinden, da
            die Qulität der Produkte einfach stark von den Versprechungen abweicht.
            \newline
            Genauso zeigt die Platform aber auch wie wichtig die eigene Erwartungshaltung
            an das Produkt ist. Jemand der auf die Platform stößt und bereits 
            \"{}auf wish bestellt\"{} als weit verbreitetes Synonym für qualitativ schlecht 
            durch das Internet kennt wird auch genau diese schlechte Qualität erwarten.
            Dieser Kunde wird dann auch zufrieden sein, wenn er sein schlechtes Produkt 
            für billigen Preis bekommt.
            \newline
            Für hohe Kundenzufriedenheit sollte deswegen immer der Grundsatz gelten:
            Nur das Versprechen, was man auch halten kann.
        \end{abstract}
    \subsection{Kundennutzen}
        \subsubsection*{1. Quelle:}
        \begin{abstract}
            %1. Quelle Kundennutzen
            \noindent "Der Nutzen ist folglich der für den Kunden in Euro ausgedrückte Wert des rele-
            vanten Vorteils. In seiner Summe ist der Nutzen das Gegenstück zum Preis bzw.
            zur Investition und hat mehrere Ebenen sowie Dimensionen. In Abb. 4.1 sind die
            drei Nutzenarten Produktnutzen, Anwendungsnutzen und Zusatznutzen dargestellt." \cite{Menthe2018}
        \end{abstract}
        \subsubsection*{2. Quelle:}
        \begin{abstract}
            %2. Quelle Kundennutzen
            \noindent 
            \"{}Ein Nachfrager wird seinen Bedarf tendenziell mit denjenigen Produkten decken 
            (im vorliegenden Buch subsumieren wir unter dem Begriff Produkt physische Produkte und Dienstleistungen),
            die seine Bedürfnisse am besten befriedigen können und somit den höchsten Nutzen (Kundennutzen)
            aufweisen (vgl. zur Diskussion des Kundennutzens Abschn. 10.1.2)." \cite{Homburg2020}
        \end{abstract}
        \subsubsection*{Eigene Erläuterung:}
        \begin{abstract}
            %Eigene Erläuterung Kundennutzen
            \noindent Der Kundennutzen gibt den Grad der Bedürfnisbefriedigung für den Kunden an.
            \newline
            Dies wird beispielsweise sehr gut an dem Vergleich von Verbrenner und E-Autos 
            deutlich.
            \newline
            Person A arbeitet bei einem Unternehmen nahe des eigenen Wohnsitzes. Dort 
            werden kostenfreie E-Ladesäulen angeboten. Neben der Fahrt zur Arbeit wird 
            das Auto nur für kleinere Einkäufe benötigt.
            Ein kleines E-Auto bietet hier hohen Nutzen. Die geringe Reichweite wird keine Rolle 
            spielen, der Treibstoff ist quasi gratis und viel Stauraum wird auch nicht benötigt.
            \newline
            Person B ist ein Bauunternehmen, welches primär Montagearbeit betreibt. Für dieses 
            Unternehmen ist das gleiche Produkt total ungeeignet.
            Strecken zum Montageort könnten aufgrund geringer Reichweite nicht in einer Fahrt zurückgelegt
            werden. Der geringe Stauraum macht den Transport von Material unmöglich.
        \end{abstract}
    \subsection{Kundenvorteil}
        \subsubsection*{1. Quelle:}
        \begin{abstract}
            %1. Quelle Kundenvorteil
            \noindent "Die Effektivität wird im Rahmen dieser Perspektive als ein externes 
            Leistungsmaß angesehen, das angibt, ob ein Unternehmen den Erwartungen und
            Ansprüchen der Kunden gerecht wird. Die Effizienz soll hingegen ein 
            internes Leistungsmaß darstellen. Es soll das Verhältnis zwischen Input und 
            Output angeben. Effektivität und Effizienz stellen im Rahmen dieser Betrachtung
            zwei Komponenten eines Wettbewerbsvorteils dar. Diese Zweidimensionalität 
            soll zeigen, dass bei der Orientierung eines Anbieters im Wettbewerb zwei 
            Richtungen der Vorteilsfindung getrennt werden
            müssen: Die anbieterexterne Sphäre und die anbieterinterne Sphäre. Vorteile,
            die aus der externen Sphäre resultieren, werden als Kundenvorteile bezeichnet.
            Vorteile, die aus der internen Sphäre resultieren, können als Anbietervorteile
            bezeichnet werden.$^{400}$ Natürlich gibt es neben dieser Definition für
            das Konstrukt des Wettbewerbsvorteils auch eine Vielzahl anderer Definitionen.$^{401}$
            Die hier getroffene Abgrenzung erscheint jedoch für die Betrachtung
            von Investitionsgütermärkten von besonderer Bedeutung." \cite{Olbrich2022}
        \end{abstract}
        \subsubsection*{2. Quelle:}
        \begin{abstract}
            %2. Quelle Kundenvorteil
            \noindent "Jedes Unternehmen tritt in seinen Marktsegmenten gegen einen oder mehrere Wett-
            bewerber an. In dieser Situation reicht es nicht aus, ausschließlich nutzenorientiert
            zu argumentieren. Neben den reinen Kundennutzen muss vielmehr der 
            Kundenvorteil treten. Der Kundenvorteil ist der Vorteil, den der Kunde beim Erwerb
            der Leistung gegenüber der des Wettbewerbers hat. Wer überlegenen Nutzen
            (= Kundenvorteil) bieten will, muss die Bedürfnisse, Probleme, Ziele und 
            Nutzenvorstellungen des Kundenunternehmens sowie die Vor- und Nachteile bzw. Stärken
            und Schwächen seines Leistungsangebotes gegenüber denen des Wettbewerbs 
            kennen. Die Positionierung zielt also auf die Optimierung des Kundenvorteils ab." \cite{Lippold2019}
        \end{abstract}
        \newpage
        \subsubsection*{Eigene Erläuterung:}
        \begin{abstract}
            %Eigene Erläuterung Kundenvorteil
            \noindent Der Kundenvorteil ist die Summe der Punkte, die ein Unternehmen besser umsetzt als ein
            Mitwettbewerber. 
            \newline
            Man betrachte 2 Versicherungsunternehmen, welche identische Lebensverischerungen anbieten.
            Wenn bei Unternehmen A der Versicherungsbeitrag 5\% höher ist als bei Unternehmen B 
            hat Untenehmen B bezüglich des Preises einen Kundenvorteil.
            \newline
            Zu beachten ist dabei, das der Kundenvorteil nur so klar fomuliert werden kann, wenn wirklich 
            alle anderen Kompenenten identisch sind. Würde B eine andere Versicherung anbieten, die komplett 
            andere Fälle absichert macht obiger Vergleich natürlich keinen Sinn.
        \end{abstract}
    \subsection{Marketingstrategie}
        \subsubsection*{1. Quelle:}
        \begin{abstract}
            %1. Quelle Marketingstrategie
            \noindent "Unter Marketingstrategien versteht man 
            Handlungsprogramme zur Erreichung von bestimmten Zielen.$^{14}$ Entsprechende
            Strategien berücksichtigen die Wettbewerbssituation, die Bedürfnisse der
            Nachfrager und das bisherige Angebot des Unternehmens. Sie führen unter
            Heranziehung von Prognosen hinsichtlich veränderlicher Umweltgrößen zu
            einer konkreten Ausprägung der Marketinginstrumente." \cite{Olbrich2022}
        \end{abstract}
        \subsubsection*{2. Quelle:}
        \begin{abstract}
            %2. Quelle Marketingstrategie
            \noindent "Marketingstrategien legen den Weg fest, wie die strategischen Marketingziele eines
            Unternehmens zu erreichen sind. Sie geben die mittel- bis langfristigen Schwerpunkte in
            der Marktbearbeitung des Unternehmens wieder, insbesondere im Hinblick auf das
            Verhalten gegenüber Kunden, Absatzmittlern und der Konkurrenz.
            Mit Hilfe der Marketingstrategie beabsichtigt das Unternehmen, die 
            Marketingproblemstellung zu lösen, um dadurch die Marketingziele zu erreichen." \cite{Bruhn2019}
        \end{abstract}
        \subsubsection*{Eigene Erläuterung:}
        \begin{abstract}
            %Eigene Erläuterung Marketingstrategie
            \noindent Die Marketingstrategie ist die Strategie, die einen Rahmen für die Umsetzung von mittel- und 
            langfristigen Marketingzielen setzt. 
            \newline 
            So kann z.B. langfristig festgelegt werden, das Werbung lediglich online oder offline stattfinden soll.
            Auch die Unterscheidung zwischen dem Wunsch internationale und national zu agieren fällt in diese Kategorie.
            Längerfristige Marketingstrategien können auch bei neuen Unternehmen, die sich etablieren wollen und dementsprechend
            ihre Produkte zunächst billig anbieten, beobachtet werden. 
            Bei solchen Strategien können auch kurzfristig (in dem Fall preispolitisch) schlechte Entscheidungen getroffen,
            welche dann trotzdem noch dem Sinn des langfristigen Ziels entsprechen. 
        \end{abstract}
        \newpage
    \subsection{Produktpolitik}
        \subsubsection*{1. Quelle:}
        \begin{abstract}
            %1. Quelle Produktpolitik
            \noindent "Die Produktpolitik beschäftigt sich mit sämtlichen Entscheidungen, die in 
            Zusammenhang mit der Gestaltung des Leistungsprogramms einer Unternehmung
            stehen und das Leistungsangebot (Sach- und Dienstleistungen) eines Unternehmens repräsentieren.
            In der Literatur hat sich der Begriff „Produktpolitik“ eingebürgert, obwohl eine eindeuti-
            ge Definition des Produktbegriffes nicht vorherrscht und damit sowohl materielle 
            (Sachgüter) als auch immaterielle Leistungen (Dienstleistungen) angesprochen sind." \cite{Bruhn2019}
        \end{abstract}
        \subsubsection*{2. Quelle:}
        \begin{abstract}
            %2. Quelle Produktpolitik
            \noindent "Die Produktpolitik steht im Zusammenhang mit allen Entscheidungen
            im Hinblick auf das gegenwärtige bzw. zukünftige Produktangebot. Bezugsobjekte
            der Produktpolitik sind sowohl Produktinnovationen als auch bereits am Markt
            etablierte Produkte. Nicht nur die Produkte selbst, sondern auch die Wahrnehmung
            der Produkte durch die Kunden sind wichtig für den Erfolg des Unternehmens.
            Diese Wahrnehmung kann im Rahmen des Markenmanagements als Teilbereich der
            Produktpolitik gezielt durch das Unternehmen gesteuert werden." \cite{Homburg2020}
        \end{abstract}
        \subsubsection*{Eigene Erläuterung:}
        \begin{abstract}
            %Eigene Erläuterung Produktpolitik
            \noindent Die Produktpolitik ist der Entscheidungsrahmen bezüglich der Produkte, die
            ein Unternehmen produziert. Es ist die Frage welche Produkte bereits existieren, wie sie 
            bezüglich der Kundenwünsche verbessert werden und welche Produkte in Zukunft dazu kommen oder
            aus dem Sortiment entfernt werden sollen mit dem Ziel sich von Mitwettbewerbern abzusetzen.
            Außerdem besteht auch die Frage nach der Wahrnehmung der Produkte. 
            \newline
            So kann z.B. ein Schokoladenhersteller entscheiden wie sein Sortiment in Zukunft aussehen soll.
            Die Nussschokolade verzeichnet Einkaufseinbrüche und wird deswegen aus dem Sortiment entfernt.
            An ihrer Stelle kommt eine neue Sorte mit Bananengeschmack. Durch das Nutzen einer neuen umweltfreundlicheren 
            Verpackung soll in Zukunft außerdem die Wahrnehmung der Kunden positiv beeinflusst werden.
        \end{abstract}
    \subsection{Preispolitik}
        \subsubsection*{1. Quelle:}
        \begin{abstract}
            %1. Quelle Preispolitik
            \noindent "Die Preispolitik als Teilbereich des Marketing stellt somit nicht ein
            isoliertes Entscheidungsfeld dar, sondern muss im Kontext des gesamten 
            unternehmerischen Handelns gesehen werden. Es bestehen z. B. Interdependenzen 
            zu den Bereichen Produktion und Finanzierung. In der Produktion muss
            die Kapazitätsplanung mit der Preispolitik koordiniert werden. Mit Blick auf
            die Finanzierung ist ein Preiskampf nur dann durchführbar, wenn die Zahlungsfähigkeit 
            des Unternehmens sichergestellt werden kann.Die Preispolitik betrifft allerdings nicht nur Fragen über die Höhe des Preises,
            es muss auch über die Form der Preissetzung entschieden werden. In der Praxis 
            können neben sogenannten linearen Preisen (fester Verkaufspreis pro
            Mengeneinheit) z. B. auch nicht-lineare Tarife und Preisbündelungen beobachtet 
            werden. Nicht-lineare Tarife beinhalten nach Verkaufsmengen gestaffelte 
            Preise oder eine Teilung des Preises in eine Grundgebühr und ein
            mengenabhängiges Entgelt – wie es in der Telekommunikationsbranche und
            bei vielen Versorgungsunternehmen üblich ist. Bietet ein Unternehmen eine
            Kombination von Produkten oder Dienstleistungen zu einem Preis an, so
            spricht man von Preisbündelung. I. d. R. verlangt das Unternehmen für dieses
            ‚Set‘ einen geringeren Preis als die Summe der Einzelpreise. Beide Maßnahmen
             sollen die Kunden dazu veranlassen, einen höheren Umsatz (pro Ge-
            schäftsvorfall) mit dem Unternehmen zu tätigen. Im ersten Fall (nicht-lineare
            Tarife) sollen die Kunden eine größere Menge, im zweiten Fall (Preisbündelung) 
            weitere Produkte kaufen." \cite{Olbrich2022}
        \end{abstract}
        \subsubsection*{2. Quelle:}
        \begin{abstract}
            %2. Quelle Preispolitik
            \noindent "Die Preispolitik beschäftigt sich mit der Festlegung der Art von Gegenleistungen,
            die die Kunden für die Inanspruchnahme der Leistungen des Unternehmens entrichten.
            Sie umfasst die Bestimmung und das Aushandeln von Preisen und sonstigen Kauf- und
            Vertragsbedingungen. Da es bei der Preispolitik nicht ausschließlich um die Preishöhe,
            sondern auch um weitere Bedingungen (z.B. Zahlungs- und Lieferbedingungen, preisähnliche 
            Maßnahmen wie Rabatte, Boni und Skonti u.a.) geht, die mit einer Leistungsinanspruchnahme 
            verbunden sind, wird sie auch als Kontrahierungspolitik bezeichnet." \cite{Bruhn2019}
        \end{abstract}
        \subsubsection*{Eigene Erläuterung:}
        \begin{abstract}
            %Eigene Erläuterung Preispolitik
            \noindent Die Preispolitik ist die Gesamtheit der Maßnahmen, die den Preis bzw. die Bezahlungmethoden 
            eines Produktes beeinflussen.
            \newline
            Ein gutes Beispiel für intensiv betriebende Preispolitik findet man bei jeglichen Internet-Abo-Anbietern.
            Sie bedienen sich an Gratis-Probemonaten, verschiedenen Zahlungsmöglichkeiten (Paypal, Kreditkarte, ...) und
            verschiedenen Abos (monatlich, jährlich) mit unterschiedlichen Kosten bzw. Rabatten. 
        \end{abstract}
    \subsection{Kommunikationspolitik}
        \subsubsection*{1. Quelle:}
        \begin{abstract}
            %1. Quelle Kommunikationspolitik
            \noindent \"{}Allgemein umfasst die Kommunikationspolitik die Gestaltung der auf die
            Märkte gerichteten Informationen und der Informationskanäle. Bei vielen
            Verbrauchern ist – nicht zuletzt aufgrund der Nutzung der Kommunikationspolitik 
            zur Differenzierung und der angestiegenen Anzahl entsprechender
            Werbebotschaften in Printmedien, im Fernsehen und im Internet – ein stark
            nachlassendes Informationsinteresse zu konstatieren. Besonders im Rahmen
            der TV-Werbung ist ein ausgeprägtes Reaktanzverhalten (z. B. bei Werbeeinblendungen) 
            festzustellen, das sich durch sogenanntes ‚Zapping’ (also ,Zapping‘
            durch Wechseln des Programmes) manifestiert. Allerdings lässt sich auch bei
            Auftreten eines derartigen Reaktanzverhaltens durch Anwendung des kommunikationspolitischen
            Instrumentariums gegensteuern. So können Produkte
            im Rahmen des Product Placement z. B. in Spielfilmen, Shows und Spielen Product Placement
            platziert werden." \cite{Olbrich2022}
        \end{abstract}
        \newpage
        \subsubsection*{2. Quelle:}
        \begin{abstract}
            %2. Quelle Kommunikationspolitik
            \noindent "Kommunikationspolitik beschäftigt sich mit der Gesamtheit der Kommunikationsinstrumente 
            und -maßnahmen eines Unternehmens, die eingesetzt werden, um
            das Unternehmen und seine Leistungen den relevanten Zielgruppen der Kommunikation 
            darzustellen und/oder mit den Anspruchsgruppen eines 
            Unternehmens in Interaktion zu treten. Die Kommunikationspolitik umfasst Maßnahmen der marktgerichteten, 
            externen Kommunikation (z.B. Anzeigenwerbung), der innerbetrieblichen, internen Kommunikation
            (z.B. Mitarbeiterzeitschrift, Intranet) und der interaktiven Kommunikation zwischen
            Mitarbeitenden und Kunden." \cite{Bruhn2019}
        \end{abstract}
        \subsubsection*{Eigene Erläuterung:}
        \begin{abstract}
            %Eigene Erläuterung Kommunikationspolitik
            \noindent Kommunikationspolitik ist die Gesamtheit aller Mittel, die verwendet werden können um Kunden 
            Produkte zu präsentieren. Weiterhin umfasst sie internen Kommunikationsstrukturen.
            \newline
            Interne Kommunikationspolitik könnte dabei z.B. durch regelmäßige Meetings
            oder Betriebsfeiern verbessert werden.
            \newline
            Externe Kommunikationspolitk kann beispielsweise das Einrichten eines FAQ's oder das Schalten 
            von Werbung für ein neues Produkt sein.
        \end{abstract}
    \subsection{Vertriebspolitik}
        \subsubsection*{1. Quelle:}
        \begin{abstract}
            %1. Quelle Vertriebspolitik
            \noindent "Die Vertriebspolitik beschäftigt sich mit sämtlichen Entscheidungen, die sich auf
            die direkte und/oder indirekte Versorgung der Kunden mit materiellen und/oder
            immateriellen Unternehmensleistungen beziehen." \cite{Bruhn2019}
        \end{abstract}
        \subsubsection*{2. Quelle:}
        \begin{abstract}
            %2. Quelle Vertriebspolitik
            \noindent "Die Vertriebspolitik umfasst Entscheidungen über marktgerichtete
            akquisitorische und über vertriebslogistische Aktivitäten. Die akquisitorischen
            Aktivitäten zielen auf die Gewinnung von Kunden und die Generierung von Kaufabschlüssen
            ab. Wesentliche Entscheidungen betreffen hier die Gestaltung des
            Vertriebssystems, die Gestaltung der Beziehungen zu Vertriebspartnern und Key
            Accounts sowie die Gestaltung der Verkaufsaktivitäten. Die vertriebslogistischen
            Aktivitäten beziehen sich auf die Gestaltung der physischen Waren und umfassen beispielsweise 
            Aktivitäten wie Lagerhaltung und Transport" \cite{Homburg2020}
        \end{abstract}
        \subsubsection*{Eigene Erläuterung:}
        \begin{abstract}
            %Eigene Erläuterung Vertriebspolitik
            \noindent Die Vertiebspolitik umfasst alle Maßnahmen, die den Weg eines Produktes vom Produzenten 
            zum Kosumenten beeinflussen.
            \newline
            Dazu zählt z.B. das Einrichten einer neuen Lagerhalle an einem neuen Standort, weil die Lieferung an Kunden 
            beschleunigt werden soll. Auch das Erschließen neuer Vertriebswege über z.B. Onlineshops oder Einkaufsmärkte
            fällt unter die Vertriebspolitik.
        \end{abstract}
        \newpage
    